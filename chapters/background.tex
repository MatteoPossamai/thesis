\chapter{Background}
\label{cha:background}

In this section, there will be all the background information relevat to the thesis,
that are needed in order to understand the matter of the project, the problem,
the technologies and the solution.

\section{CPU speculation contract}
\label{cha:CPU speculation contract}

\section{Revizor}
\label{cha:Revizor} Revizor is an open source fuzzing tool developed by Microsoft
Research. It has been developed to detect black-box CPU leakeages, just starting
from a CPU speculation contract (\ref*{cha:CPU speculation contract}). Then, it generates
a number of test cases, executing therefore some fuzzing to detect evantual
leakeages in the contract. Once the test cases has been generated, they are
runned on the CPU, and then Revizor looks at the information that are leaked,
and then compare them with the CPU contract. With this information, we are able
to understand if the CPU is leaking information. \\

The researchers tested Revizor on different x86 Intel CPUs, finding some known vulnerabilities,
such as Spectre, MDS (Microarchitectural Data Sampling) and LVI (Load value
injection), and some novel ones. All this is done in a short amount of time, in
an automated fashion. The tool is mainly written in Python, with some C code for
performance improvements. It is available on GitHub, and can be used by anyone
to test any CPU. \\

In this project, Revizor is used as part of the toolchain, to detect the
contract of a CPU. starting from our current CPU leakeage contract, we run
Revizor on this specification, and then we check if we obtain some violations.
If so, we add the discovered violations to the contract, enriching it with novel
information about the CPU. And then we loop over. The loop will be futher
discussed in Section \ref*{cha: Loop outlook}. \\

Further information about Revizor can be found: \\ Overall research paper:
\url{https://www.microsoft.com/en-us/research/publication/revizor-testing-black-box-cpus-against-speculation-contracts/}
\\ GitHub Repository \url{https://github.com/microsoft/sca-fuzzer/} \\Documentation:
\url{https://microsoft.github.io/sca-fuzzer/}

\section{DSL}
\label{cha:DSL} To represent the CPU contract, we needed a language that was
able to describe the ISA of a CPU and its contract. To archieve this need, at
the beginning of the project, a Domain Specific Language (DSL) was developed.
This language is able to define, check and evaluate a number of boolean conditions,
such as AND, OR, NOT, EQUAL, and to access CPU's registers. This was enough as a
proof of concept to see wheather the whole system was working, but now with the
further steps of the project, this is not enough anymore. To have a more
research-validated, robust, complete and universal language, the team decided to
switch from this DSL to using the BIR language (more in Section \ref*{cha:BIR}).

\section{Rosette}
\label{cha:Rosette}

\section{Loop outlook}
\label{cha: Loop outlook}

\section{BIR}
\label{cha:BIR}