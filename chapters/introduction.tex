\chapter{Introduction}
\label{cha:introduction}

Micro-architectural Leakage Contract Synthesis (Malcos) is a synthesis engine
for generating contracts representing microarchitectural leakages on CPUs. It is
currently under development of a research team made with researchers from University
of Trento, Imdea and KTH. The goal of this project is to provide a tool that can
synthesize speculation contracts of black box CPUs. This is done to then, in futher
steps, be able to inspect vulnerabilities of CPU and correctly and automatically
patch them. The tool is written mainly in Python and Racket, using the Revizor (\cite{article},\cite{repo},
\cite{misc}) tool and the Rosette language.

At the beginning of the project, the team develped and used a Domain Specific
Language (DSL) to write the specifications of the CPU contracts. But once the project
started to grow, they saw that the language was not the right language for the
project. The DSL was not in fact powerful enough, to represent in a platform-independent
way all CPU possible contract. A new semantic was therefore needed. Also, a group
of researchers that was working on similar topics, developed a new language, called
Binary Intermediate Representation (BIR, \cite{bir_pub}), which they created with
the goal of representing the ISA of the CPU in a platform-independent, powerful
and complete way, to have an universal language for this purpose. This situation
created the necessity to switch the current DSL to BIR.

To do so, all the current DSL pipeline needed to be replaced by a new one,
suited for the BIR language. This would have been the best choice seens translating
from the DSL to BIR and viceversa would not have been the best choice, since BIR
is a more complex and powerful language. It was not automatic to complete all the
translation.

Therefore, in this thesis we implement all the new pipeline and logic needed to
handle the BIR language in this particular project, and we validate the correctness
of the final outcome. To be more procise, in this thesis we will implement the parser
for the language, the tracer that interpretes the contract and populates the
trace and the engine for generating the new synthesized contracts.

Also, we thought about whether it would have been worth migrating the pipeline,
or part of it from the native Python, to Rust, a more performant and safe
programming language. In this thesis, we will also explore the research, the proof
of concepts (PoC) developed and the final outcome of this trial, that has been
quite surprising giving the premise.

I wrote and run multiple different tests to validate the correctness of the new
pipleline. The majority of them are unit tests, to test each step of the
pipeline and ensure correct behavior at a local scale, and multiple tests involving
all the moving part. To test the later, I used already tested contracts, and
compared the outcome with the DSL pipeline, to ensure correctness and consistency.
In total, more than 150 tests are present in the codebase contribution of this thesis.

At this point, also, the Malcos project is still under development, but now is
using the BIR language to represent the contract instead of the legacy DSL. This
further validates the work done. \\

\textbf{Outline}: This thesis is a overview of the contribution done to the Malcos
codebase, exploring all the steps taken, the choice made with their whys, and
prove the validity of the developed solution.

To do so, Section \ref{cha:background} presents background overview of all the needed
concepts to understand the work and the project we will be talking about, and
all the tools used. Section \ref{cha:core} will be the core of the thesis, with
all the steps taken to transition from the DSL to BIR, the validation of each step
and the assessment of the final results. In addition, in this chapter there will
be all the explaination of the research done to decide whether it was worth migrating
the pipeline from the native Python code to a Rust implementation.