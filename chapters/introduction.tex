\chapter{Introduction}
\label{cha:introduction}

Micro-architectural Leakage Contract Synthesis (Malcos) is a synthesis engine
for generating contracts representing speculation behavior of black-box CPUs. It
is currently under the development of a research team made with researchers from
University of Trento, Imdea and KTH. The goal of this project is to provide a tool
that can synthesize speculation contracts of black box CPUs, providing more
information about their behavior. This is done to then, in further steps, be able
to inspect vulnerabilities of CPU, to then correctly and automatically patch
them, in a platform-dependent way. The tool is written mainly in Python and
Racket, using the Revizor (\cite{article},\cite{repo}, \cite{misc}) tool and the
Rosette language\cite{ros}. As the time of this thesis, the project is not open-sourced,
but the plan is to make it public once it is completed, tested and validated.

At the beginning of the project, the team developed and used a Domain Specific
Language (DSL) to write and describe the specifications of the CPU contracts. But
once the project started to grow, they saw that the language was not the right
fit for the project. The DSL was not, in fact, powerful enough to represent in a
platform-independent way all possible CPU contracts and operations. A new
semantic was therefore needed. Also, a group of researchers of KTH institute
involved in the same topics and also collaborating with the team in other
research projects, developed a new language, called Binary Intermediate
Representation (BIR, \cite{bir_pub}), which they created with the goal of
representing the Instruction Set Architecture (ISA) of the CPU in a platform-independent,
powerful and complete way, to have an universal language for this purpose. These
two factors lead to the decision of migrating the semantic that the project used
for the contract from the DSL to BIR.

To do so, all the current DSL pipeline needed to be replaced by a new one, suited
for the BIR language. A full rebuild was needed since translating from BIR to
the DSL to then inject it into the tool, and translating the output to BIR again
was not a suitable option. This was because of the fact that BIR is far more
complex then the DSL, and therefore not all the constructs were available. This
meant making the DSL more powerful, or switch directly and entirely to BIR, as
has we did.

Therefore, in this thesis we implement all the new pipeline and logic needed to handle
the BIR language for this particular project, and we validate the correctness of
the final outcome. To be more precise, in this thesis we implement the parser
for the language, the tracer that interprets the contract and populates the trace
and the engine for generating the new synthesized contracts.

Also, we thought about whether it would have been worth migrating the pipeline, or
part of it from the native Python, to Rust, a more performing and safe programming
language. In this thesis, we also explore the research, the proof of concepts (PoC)
developed and the final outcome of this trial, that has been quite surprising giving
the premise.

We wrote and run multiple different tests to validate the correctness of the new
pipeline. The majority of them are unit tests, to test each step of the process and
ensure correct behavior at a local scale, and multiple tests involving all the
moving part. To test the latter, we used already tested contracts, and compared the
outcome with the DSL pipeline, to ensure correctness and consistency. In total,
more than 200 tests are present in the code contribution of this thesis.

At this point, also, the Malcos project is still under development, but now is using
the BIR language to represent the contract instead of the legacy DSL. This
further validates the work done.

\textbf{Outline}: This thesis is the overview of the contribution done to the
Malcos code, exploring all the steps taken, the choice made with their
motivations, and proofs the validity of the developed solutions.

To do so, Section \ref{cha:background} makes a background overview of all the
needed concepts to understand the work and the project we are talking about, and
all the tools used in this thesis. Section \ref{cha:core} presents the core of
the thesis, with all the steps taken to transition from the DSL to BIR, the
validation of each step and the assessment of the final results. Follows the additional
chapter \ref{cha:Migration to Rust} where there are all the explanation of the research
done to decide whether it was worth migrating the pipeline from the native Python
code to a Rust implementation. Follows section \ref{cha:related_works}, where we
talk about all the related works present in the academic literature at the moment,
how we took inspiration and how we compared our work with them. Section \ref{cha:conclusion}
presents the conclusion of the thesis, with some brief thoughts about the work
done and the future development of the project.