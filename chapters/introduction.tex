\chapter{Introduction}
\label{cha:introduction}

\section{Context}
\label{cha:Context}

For a lot of years, CPU vendors to increase the performance of the bare metal
have tried to increase clock frequency. In recent years, this approach has lead
to a plateau, since this requires huge power consumption, generates a lot of
heat and ultimately consume the CPU life. This plateau has been reached at around
4 GHz. \\ Regardless of this limitations, vendors have tried to improve performance
in other ways, reliying not to optimizing the metal performance, but applying smart
heuristics to the behauvior of the CPU. One of this improvement is called
speculation. \\ Speculation happens when the CPU tried to predict the outcome of
a branch instruction, and therefore start executing the logic beforehand. Once the
prediction is confirmed, the CPU can continue executing, but now has an hedge in
the execution.\\ This approach is really beneficial in terms of performance gains,
but has his drawbacks. In fact, this leads to potential security vulnerabilities,
that are exploitable with attacks such as Spectre and Meltdown.

\section{Problem}
\label{cha:problem}

\section{Solution}
\label{cha:solution}

\section{Validation}
\label{cha:validation}

\section{Outline}
\label{cha:outline}