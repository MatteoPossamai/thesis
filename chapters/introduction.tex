\chapter{Introduction}
\label{cha:introduction}

Micro-architectural Leakage Contract Synthesis (Malcos) is a synthesis engine
for generating contracts representing microarchitectural leakages on CPUs. It is
currently under development of a research team made with researchers from University
of Trento, Imdea and KTH. The goal of this project is to provide a tool that can
synthesize speculation contracts of black box CPUs. This is done to then, in futher
steps, be able to inspect vulnerabilities of CPU and correctly and automatically
patch them. The tool is written mainly in Python and Racket, using the Revizor
tool and the Rosette language.

At the beginning of the project, a Domain Specific Language (DSL) was developed and
used to write the specifications of the CPU contracts. But once the project
started to grow, the need for a more robust, flexible and researched language
started to arise. The DSL was not in fact powerful enough, to represent in a
platform-independent way all CPU possible contract. A new semantic was therefore
needed, and the choice turned out to be the Binary Intermediate Representation (BIR)
language. It is the best candidate for representing the contracts, since it has
been developed with this goal in mind by a group of researchers. They designed BIR
to be a powerful platform independent language, and become a standard in this
vertical. Using BIR would make the Malcos project more transferable and utilizable
by other research groups. \\

So, there was the need to transition from the DSL semantic to BIR. To do so, all
the current DSL pipeline needed to be replaced by a new one, suited for the BIR language.
This would have been the best choice seens translating from the DSL to BIR and
viceversa would not have been the best choice, since BIR is a more complex and powerful
language. It was not automatic to complete all the translation.

To validate the new pipeline, multiple different tests were written and run.
There has been a lot of unit tests for each step of the pipeline, to ensure correct
behavior at a local scale, and multiple tests involving all the moving part. This
latter type of tests has been done with already-tested contracts, and compared
the outcome with the DSL pipeline, to ensure correctness and consistency. In
total, more than 150 tests are present in the codebase contribution of this thesis.

Also, we thought about whether it would have been worth migrating the pipeline, or
part of it from the native Python, to Rust, a more performant and safe programming
language. In this thesis, we will also explore the research, the proof of
concepts (PoC) developed and the final outcome of this trial, that has been quite
surprising giving the premise.

At this point, also, the Malcos project is still under development, but now is
using the BIR language to represent the contract instead of the legacy DSL. This
further validates the work done. \\

\textbf{Outline}: This thesis is a overview of the contribution done to the Malcos
codebase, exploring all the steps taken, the choice made with their whys, and
prove the validity of the developed solution.

To do so, Section \ref{cha:background} there will be a background overview of all
the needed concepts to understand the work and the project we will be talking about,
and all the tools used. Section \ref{cha:core} will be the core of the thesis,
with all the steps taken to transition from the DSL to BIR, the validation of each
step and the assessment of the final results. In addition, in this chapter there
will be all the explaination of the research done to decide whether it was worth
migrating the pipeline from the native Python code to a Rust implementation.