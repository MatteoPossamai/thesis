\chapter{Related Work}
\label{cha:related_work}The paper that has been the most important for this
thesis is that of Linder, Guanciale and Metere et al. \cite{bir_pub}. This article
that We refered multiple times in this thesis, is focused mainly in the
automated verification of binary files, proving their validity and correctness.
To archive this goal, the authors created a new language, called Binary
Intermediate Representation (BIR), that could describe the Instruction Set Architecture
(ISA) of a CPU in a platform-independent way, to pursue their goals. This is the
birth of the BIR language, that has been used in this thesis. The article's writers
are also part of the same research group that we have been part of, so the
stimuli and the ideas that have been developed in this paper have been the base
of the motivation for using BIR in the Malcos code.

Also, another important piece of research for the scope of this thesis has been the
research done by Microsoft researchers, in particular in Oleksii Oleksenko, Christof
Fetzer, Boris Köpf, Mark Silberstein et al. \cite{article}. This article is where
the Revizor tool has been presented, and some vulnerabilities of some CPUs were
discovered. Since Revizor is a crucial part of this project as a whole, and
something we had the opportunity to touch with our works, it was worth reading this
article to better understand the tool and its capabilities, to then also
understand better how the Malcos project used it.

Once last relevant piece of research that has been important for this thesis is
Marco Guarnieri, Boris Köpf, Jan Reineke, and Pepe Vila. at el. \cite{contracts_paper}.
In this paper, the authors describe the hardware and the software contracts that
are required for secure speculation. In this paper, the concepts of contracts and
speculation could be better understood. This paper has been important for the
thesis, since it allowed us to fully understand some concepts around the speculation
and contracts. This is also a foundation paper for the Malcos project, since Marco
Guarnieri, one of the author, is one of the main contributors to the Malcos project,
and has also been one on the Revizor code.

\chapter{Conclusion}
\label{cha:conclusion}
\section{Considerations}
\label{cha:Considerations} This focus of this thesis is the development of the pipeline
for being able to use the BIR language in the Malcos project. This is because
the legacy language, the DSL, was not inadequate for the needs and scale of the
project. To then provide a more research-based and robust solution, this
transition was necessary.

In the end, the switch from the legacy DSL to the BIR language was successful,
in the sense that we validate the pipeline, as described in the previous chapters.
The more complex language that was introduced allowed for more powerful
synthesis of contract, but also took a slight performance hit, inherent to the
complexity of the language. With respect to whether this switch was actually worth
it, is hard to assess a priori. The Malcos project, even with all his ambitions,
has not reached a stage where the contracts are actually so complicated, and so
the DSL could have been enough. Even though the project is supposed to grow a lot
in the future and being part of a bigger tool chain, and therefore be able to
synthesize more complex contracts, for example real-world CPU ones, having a
more robust and complete language seemed a reasonable choice. We have to bare in
mind, though, that we do not know how actually even the most complex contracts looks
like, so we do not know if BIR is actually too complex, or even to simple for the
task, and if the performance will be a bottleneck in the future.

It seems unreasonable that BIR is not powerful enough, since it is able to represent
all the CPU operations, and therefore should be able to represent any contract, unless
new concepts are discovered in further research. On the other hand, the possibility
that BIR is too complex is more likely, given his completeness and
expressiveness. This could backfire, since BIR can be harder to read, since more
verbose than the legacy DSL, minimizing therefore the readability of the
contracts by humans.

In the end, though, both the cases in which BIR is not the best fit are quite
remote, so this switch was worth the effort, and it is better that has been done
at early stage of the project, where all the tooling is not yet fully developed,
otherwise the re-factor could have been larger and more painful.

Regardless of the actual final necessity or not, the switch came with the cost of
a slight performance hit, given the complexity and recursive nature of the BIR
language. This is the feature that allows it to be so powerful and expressive,
but also that makes it harder and slower to generate with Rosette.

It is also worth considering that Malcos is not the only project in mind of the
reseach group. It could just be the first step of a bigger tool chain for security
analysis at CPU binary level. Therefore, for future works, detailed in section
\ref{cha:Future work}, the switch to BIR could be a crucial step, and definetly
worth the effort.

\section{Future work}
\label{cha:Future work} This project, overall, has a huge potential for
different use cases an research paths. The first thing that seems to have the higher
potential is to, once the contract of the CPU has been synthesized, we know all the
potential vulnerabilities that could be exploited. Using this pointer, a good use
of this information would be to create some compiler-level checks or modification
to the target code so that the code becomes actually secure. This possible
evolution has huge potential, since this could almost eliminate the risks that speculation
brings, and only leaving the improvements that it has to offer. This could be a huge
step forward in the world of security, and could also interest a lot of vendors.

An eventual alternative could be to patch the code afterwards, even though this seems
to be a less appealing goal, since this would add another step in the
compilation process, making it slower. And for large code this would be a
bottleneck, while directly writing the compiled code with the possible security flaw
in mind could yield better results in term of compilation performance.

Eventually, the knowledge of the contract of the CPU could be an information
that could help in the discover of new vector of attack, some other alternative to
Specter as Revizor's team discovered. Also this could let room for improvement
in the security of the speculation practice, allowing for secure increase in performance.

Overall, every possible future work seems to have potential, and also remain in the
domain of security and low level programs.